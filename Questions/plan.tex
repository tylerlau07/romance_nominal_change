\documentclass{article}
\title{Questions for Romanian-Slavic Project}
\author{Tyler Lau}
\date{}

\usepackage{fullpage}
\usepackage{array}

\newcolumntype{L}[1]{>{\raggedright\let\newline\\\arraybackslash\hspace{0pt}}m{#1}}
\newcolumntype{C}[1]{>{\centering\let\newline\\\arraybackslash\hspace{0pt}}m{#1}}
\newcolumntype{R}[1]{>{\raggedleft\let\newline\\\arraybackslash\hspace{0pt}}m{#1}}

\begin{document}

\maketitle

\begin{tabular}{L{3cm} | L{6cm} | L{6cm} }
{\bf Question}	&	{\bf Prediction}	&	{\bf Importance} \\ \hline
What happens to gender from Vulgar Latin to Romanian?	& The phonological changes from Vulgar Latin to Romanian led to syncretism of gender endings and confusion of the neuter. Alternatively, the gender system was influenced by the Slavic one, but this hypothesis appears to be rejected by Petrucci. & A careful look will provide us with one case study of how a gender system can change diachronically. \\ \hline
When words are borrowed (Slavic, Dacian/Albanian), how do they behave w.r.t. gender? & The gender in the source language will be preserved regardless of phonological shape (cf. Sp./Port. borrowings from Greek) & We can use this question to explore whether the gender in the source language trumps phonological shape (and maybe semantic grouping). If we look at this diachronically (ex. old borrowings from Dacian), we may see that phonological shape triumphs over source language over time, which tells us that there is competition in language acquisition. \\ \hline
Even if words are not borrowed, is the gender of words in Romanian affected by the gender of the Slavic words? & Most likely it is not, as the lexical form itself is not borrowed. However, it is possible that semantic groups that are associated with certain genders are shaped by contact with Slavic. & This question has implications for bilingualism and its effect on semantic classification for the purposes of gender and whether there is convergence. \\ \hline
Are singulars and plurals treated as the same or separate entries in the lexicon? & Difficult to predict. Morphology tells us they should be treated the same, but the Romanian gender system suggests that being treated as separate lexical forms makes gender assignment more intuitive. When modeling the diachronic change, we can see whether the model is more accurate when the forms are treated separately or when they are treated together & Implications for morphology and for the acquisition of words that are morphologically related. Does this morphology trump the efficiency of memorization? \\ \hline
How does gender change naturally? Is there more preservation in low contact areas (such as Faroese, perhaps)? & Gender likely will be more constant without contact (though this may be unclear too--for example, Sp. and Port. are close to each other genetically and areally, but even basic words are sometimes opposing in gender. {\sl nariz} 'nose' is masc. in Port. but fem. in Sp. & Implications for general processes of language change and what features remain stable over time (none or few studies on whether gender does) as well as whether feature is easily or not so easily affected by contact.

\end{tabular}

\end{document}